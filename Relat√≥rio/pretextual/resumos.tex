% ---
% RESUMOS
% ---

% RESUMO em português
\setlength{\absparsep}{18pt} % ajusta o espaçamento dos parágrafos do resumo
\begin{resumo}
A atitude de um veículo refere-se à orientação dos eixos de coordenadas fixos no corpo em relação a um referencial definido e o controle de atitude, por sua vez, são as técnicas para manter a orientação do corpo dentro dos limites definidos. Dentro das técnicas conhecidas, têm-se a modelagem Newton-Kepler para dinâmica orbital, a de Newton-Euler para rotação de corpos rígidos e o controle proporcional inercial derivativo para o controle de atitude. O presente trabalho de graduação com o objetivo de atestar o aprendizado apresenta o estudo, modelagem e análise da mecânica orbital e rotacional abordando o controle de atitude de um CubeSat em órbita circular. Ao decorrer do trabalho será feita comparação com a literatura e apontamento de discrepâncias e proposta de abordagem.

 \textbf{Palavras-chaves}: CubeSat, Controle de Satélites, Dinâmica e Controle de Veículos Espaciais, Controle PID, Dinâmica Orbital.
\end{resumo}

% ABSTRACT in english
\begin{resumo}[Abstract]
 \begin{otherlanguage*}{english}
   This is the english abstract.

   \vspace{\onelineskip}
 
   \noindent 
   \textbf{Keywords}: latex. abntex. text editoration.
 \end{otherlanguage*}
\end{resumo}