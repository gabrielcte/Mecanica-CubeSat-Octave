\chapter{Materiais e Métodos}\label{cap:ferramentas}

\lipsum[12]
\section{Sistemas de Referência Orbita, Corpo e Inercial Equatorial Centrado na Terra}\label{sec:Sistemas de Referência}

O sistema de coordenadas geocêntrico-equatorial é escolhido como referencia inercial e global para o presente trabalho. Como o intervalo de estudo do fenômeno é irrelevante ao tempo de modificação desse sistema de referência tem-se como hipótese simplificadora que esse é considerado um sistema não rotativo, assumido como fixo no espaço.

\section{Fomulação de Newton-Kepler para Órbita}
\lipsum[12]

\section{Dinâmica Rotacional de Corpo Rígido de Newton-Euler}
\lipsum[12]

\section{Controle Proporcional, Integral, Derivativo}
\lipsum[12]

\section{Integração Numerica}
\lipsum[12]

\section{Octave}
\lipsum[12]

\section{Matlab}
\lipsum[12]

\section{Simulink}
\lipsum[12]
