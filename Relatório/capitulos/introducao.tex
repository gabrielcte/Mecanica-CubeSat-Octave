\chapter*[Introdução]{Introdução}
\addcontentsline{toc}{chapter}{Introdução}

O acúmulo do avanço técnico-científico até o final do século XIX permitiu que a humanidade  superasse o caráter de espectador para protagonista na exploração do espaço. Quando adentrado o século XX essa exploração evolui, se tornando fundamental para a evolução da sociedade e soberania das nações. Foi naquele século que as técnicas foram desenvolvidas para acompanhamento e estruturação das missões, tripuladas ou não, que alcançaram e estudaram nossa vizinhança. Do cenário belicoso da Guerra Fria frutifica os programas Sputnik, Explorer, Vostok, Gemini, Apollo etc, que caracterizam, uma “era de ouro”, a corrida espacial compreendida entre 1953 a 1975.

Com  o final da Guerra Fria se inaugura uma cultura de cooperação entre as diversas agências espaciais do mundo, e adiciona a participação cada vez mais presente do setor privado. E é nesse novo contexto que surge uma nova corrida espacial, com várias faces: exploração de recursos espaciais, presença de longo prazo em estações e corpos celestes e exploração de Marte. Na transição dos séculos já são fundados dois grandes nomes do setor privado espacial, a Blue Origin (2000) e a SpaceX (2002). E em menos 15 anos, em 2016, Elon Musk fundador, CEO e CTO da SpaceX estava apresentando no 67º Congresso Internacional Astronáutico “Interational Astronautical Congress (IAC)”, Guadalajara - México, o plano “Mars and Beyond” com o ano de 2023 para começar os voos tripulados para Marte em seu cronograma. Em 2017 o Presidente dos Estados Unidos, Donald Trump, assina uma nova política espacial nacional, que prevê o trabalho em conjunto da Nasa com setores privados, com o objetivo de retornar à Lua e explorar Marte, programa batizado em 2019 de Artemis.

1972 a Apollo 17 foi a última missão tripulada à Lua, após, a presença humana no espaço teve como limite a órbita terrestre baixa, protegida pelo campo magnético da Terra. O espaço externo ao escudo de proteção magnético que a Terra oferece, é danoso tanto para equipamentos eletrônicos quanto para organismos vivos. De onde emergem os desafios para a exploração de Marte, assentamento na Lua e mineração de corpos celestes, tais desafios fomentam a seguinte estratégia: estudo do ambiente espacial por missões não tripuladas de baixo custo.

Os CubeSats por suas características de: modularidade, uso de componentes de prateleira menor custo e difusão nas diversas agências espaciais, institutos de pesquisa e educação ao redor do mundo. O torna a solução imediata lógica para missões além da órbita baixa terrestre com o objetivo de estudar o ambiente espacial, testar componentes, testar técnicas e métodos e mapear asteroides, a Lua e Marte. Assim, para a demonstração de tecnologia, foram desenvolvidas os CubeSats gêmeos de comunicação MarCo A e B, com o objetivo de demonstrar a capacidade de comunicação com a sonda terrestre Insight e a Terra.  Outro exemplo é a missão Artemis I, na qual a NASA ofereceu carregar treze CubeSats 6U para missões lunares e exploração do espaço profundo.

Desde 2014, têm-se participações do Brasil no estudo e uso de CubeSats, com a demonstração da capacidade e custos da plataforma, conclui-se ser cada vez mais adequada e preferível solução para a exploração dos recursos espaciais. Em conjunto, a Agência Espacial Brasileira, os Institutos de Pesquisa e a Universidade trabalham em conjunto para dominar tal plataforma. Temos como exemplo o Nanosatc-Br, ItaSat, Serpens. E mais recentemente o Pion-BR1, primeiro satélite criado por um startup brasileira, lançado pelo foguete Falcon 9, da SpaceX.

Para auxiliar a entrada do Brasil no estudo do espaço além das órbitas baixas terrestres e acompanhar as tendências apresentadas acima, o objeto de estudo deste trabalho de conclusão de curso está inserido na grande área de Dinâmica e Controle da Engenharia Aeroespacial. Sendo ele o estudo da resposta PID de um CubeSat em uma órbita circular.

O presente estudo é dividido em seis capítulos, apresentados a seguir:

O Capítulo 1 apresenta referências utilizadas.

O capítulo 2 inicia sobre as ferramentas, técnicas e plataformas exploradoras e o grau de inovação das mesmas.  

O capítulo 3 aprofunda os conceitos relacionados ao desenvolvimento do presente estudo, os fundamentos físicos para a modelagem, as estratégias para o controle do modelo e as referências para os atuadores.

O capítulo 4 desenvolve a metodologia para a aquisição dos dados que permitem a comparação, além das ferramentas utilizadas.

O capítulo 5 apresenta de forma clara, o que está sendo proposto no estudo, métricas usadas e considerações no processo.

O capítulo 6 apresenta os resultados obtidos na simulação e discussão do mesmo.

O capítulo 7 e final, é a conclusão do projeto.
\section*{Motivação:}\label{sec:Motivação:}
\addcontentsline{toc}{section}{Motivação:}
No contexto de aluno de graduação, o presente trabalho de conclusão de curso da engenharia aeroespacial, busca auxiliar os pares do autor, com a apresentação do conteúdo referente de forma sintética, didática e centralizada. Para exploração do tema, tanto em iniciações científicas ou competições acadêmicas relacionadas.
\section*{Objetivo:}\label{sec:Objetivo:}
\addcontentsline{toc}{section}{Objetivo:}
Este trabalho tem como objetivo a síntese e aplicabilidade direta dos conhecimentos adquiridos no curso de Engenharia Aeroespacial da Universidade Federal do ABC.

O tema do estudo é dinâmica orbital e rotacional e controle PID de CubeSats. O foco neste estudo é explorar a literatura e a partir da modelagem computacional visualizar o comportamento da trajerória e atitude e a respota do controlador PID. Para isso usou-se as ferramentas de MATLAB e SIMULINK e por fim Octave na reprodução e análise de livros e trabalhos académicos.
\subsection*{Objetivo Geral:}
\begin{itemize}
\item  Reproduzir de forma computacional a mecânica rotacional e orbital de um CubeSat e da resposta do controlador PID. 
\end{itemize}

\subsection*{Objetivo Expecífico:}
\begin{itemize}
\item Modelar mecânica orbital de um CubeSat.
\item Apresentar visão 3D e tracejado de Solo da Órbita.
\item Modelar mecânica orbital de um CubeSat.
\item Apresentar gráficos do comportamento rotacional não controlado.
\item Modelar sistema de Controle PID para um CubeSat.
\item Afinação do controlador PID pelo método de Ziegler Nichols.
\item Analisar resposta encontrada pelo controlador.
\end{itemize}

